\begin{zsection}
  \SECTION expansion\_rules \\~~ \parents normalization\_rules
\end{zsection}

This section defines the unfolding of schema expressions.

The Theta rule rewrites expressions into a binding.  The type proviso
is used to calculate the signature of $E$, then the theta proviso
generates the appropriate binding expression from that signature.  For
example, the binding $\lblot x==x,y==y \rblot$ is computed for
expressions $\theta~[x,y:\nat]$.

\begin{zedrule}{Theta}
  E :: \power [D | true] \\
  \theta [D | true] \is E2
\derives
  \theta E = E2
\end{zedrule}

The next rule handles decorated $\theta$ expressions.  However, this
version does not handle lists of strokes.  In the future we will add
support for Stroke-List jokers, so that all lists of decorations can
be handled by this rule.

\begin{zedrule}{ThetaDecor}
  E :: \power [D | true] \\
  \theta [D | true]~{}^S \is E2
\derives
  \theta E~{}^S = E2
\end{zedrule}


Rule SchemaDecor works only for innermost!

\begin{zedrule}{SchemaDecor}
  [D1|P1]~{}^S \is [D2|P2] \\
\derives
  [D1|P1]~{}^S = [D2 | P2]
\end{zedrule}




\begin{zedrule}{SchemaRenaming}
   E [ RL ] \is E2 \\
\derives
   E [ RL ] = E2
\end{zedrule}

Schema composition is one of the most complex schema operators.
However, this SchemaComposition rule captures its semantics concisely.
First, the signatures of the two expressions $E1$ and~$E2$ are
computed, and all the names from $D2$ that have a primed version in
$D1$ are collected in $D5$.  Note that we currently use the rarely
used decoration $_9$.  In the future we are going to support a special
decoration that cannot be entered by users.

\begin{zedrule}{SchemaComposition}
  E1 :: \power [D1 | true] \\
  E2 :: \power [D2 | true] \\
  (split [D1|true]) \is {}\\~~ ([D3|true]? \land [D4|true] \land \\~~~~ [D5|true]' \land [D6|true]!) \\
  ([D5 | true] \schemaminus [D2 | true]) \is E3\\
  ([D5 | true] \schemaminus E3) \is E4 % matching names
\derives
  E1 \semi E2 = {}\\~~ (\exists E4~_9 @ {}\\~~~~ (\exists E4~' @ [E1 | \theta E4~' = \theta E4~_9]) \land {}\\~~~~ (\exists E4   @ [E2 | \theta E4   = \theta E4~_9]))
\end{zedrule}

\begin{zedrule}{SchemaPiping}
  E1 :: \power [D1 | true] \\
  E2 :: \power [D2 | true] \\
  (split [D1|true]) \is {}\\~~ ([D3|true]? \land [D4|true] \land {}\\~~~~ [D5|true]' \land [D6|true]!) \\
  (split [D2|true]) \is {}\\~~ ([D7|true]? \land [D8|true] \land {}\\~~~~ [D9|true]' \land [D10|true]!) \\
  ([D6 | true] \schemaminus [D7 | true]) \is E3\\
  ([D6 | true] \schemaminus E3) \is E6 % matching names
\derives
  E1 \pipe E2 = {}\\~~ (\exists E6~_9 @ {}\\~~~~ (\exists E6~! @ [E1 | \theta E6~! = \theta E6~_9]) \land {}\\~~~~ (\exists E6~? @ [E2 | \theta E6~? = \theta E6~_9]))
\end{zedrule}



The following schema construction rules replace schema operations by
schema construction expressions.

The semantics of schema negation requires schemas to be
normalised before their predicate is negated.

\begin{zedrule}{SchemaNegation}
  [D|P] :: \power [D1|true]
\derives
  (\lnot [D|P]) = [D1 | \lnot (pred [D|true] \land P)]
\end{zedrule}

The SchemaConjunction rule is straightforward.  It uses a type proviso
to check that the signatures of the two schemas are compatible.  The
schema disjunction, implication and equivalence rules are similar.

\begin{zedrule}{SchemaConjunction}
  ([D1 | true] \land [D2 | true]) :: \power [D3 | true]
\derives
  ([D1|P1] \land [D2|P2]) = {}\\~~ [D3 | {}\\~~~~ pred [D1|true] \land P1 \land {}\\~~~~ pred [D2|true] \land P2 ]
\end{zedrule}

\begin{zedrule}{SchemaDisjunction}
  ([D1 | true] \land [D2 | true]) :: \power [D3 | true]
\derives
  ([D1|P1] \lor [D2|P2]) = {}\\~~ [D3 | {}\\~~~~ (pred [D1|true] \land P1) \lor {}\\~~~~ (pred [D2|true] \land P2) ]
\end{zedrule}

\begin{zedrule}{SchemaImplication}
  ([D1 | true] \land [D2 | true]) :: \power [D3 | true]
\derives
  ([D1|P1] \implies [D2|P2]) = {}\\~~ [D3 | {}\\~~~~ (pred [D1|true] \land P1) \implies {}\\~~~~ (pred [D2|true] \land P2) ]
\end{zedrule}

\begin{zedrule}{SchemaEquivalence}
  ([D1 | true] \land [D2 | true]) :: \power [D3 | true]
\derives
  ([D1|P1] \iff [D2|P2]) = {}\\~~ [D3 | {}\\~~~~ (pred [D1|true] \land P1) \iff {}\\~~~~ (pred [D2|true] \land P2) ]
\end{zedrule}

Schema projection is also similar to schema conjunction but the
resulting schema has only the names that are declared in the second
schema.  Any other names that are declared in the first schema are
hidden existentially.

\begin{zedrule}{SchemaProjection}
  ([D1 | true] \land [D2 | true]) :: \power [D3 | true] \\
  [D2 | true] :: \power [D4 | true] \\
  ([D3 | true] \schemaminus [D4 | true]) \is [D5 | true]
\derives
  [D1|P1] \project [D2|P2] = {}\\~~ [D2 | (\exists D5 @ pred [D1|true] \land P1 \land P2)]
\end{zedrule}

To calculate the precondition of a schema, we must existentially hide
all the output names, such as $x!$ or $x'$.  The $split$ operator in
the following rule takes a declaration list and splits the names into
four groups according to their final decoration (for example, $[x, x',
y?, y!, z:T] \is [y:T]? \land [x:T; z:T] \land [x':T]\land [y:T]!$).

The declarations whose names have no decorations or a final decoration
of $?$ are collected in $D6$.

\begin{zedrule}{SchemaPrecondition}
  [D | true] :: \power [D1 | true] \\
  (split [D1|true]) \is {}\\~~ ([D2|true]? \land [D3|true] \land {}\\~~~~ [D4|true]' \land [D5|true]!) \\
  ([D2|true]? \land [D3|true]) :: \power [D6|true]
\derives
  \pre [D|P] = {}\\~~ [D6 | {}\\~~~~ (\exists ([D4|true]' \land [D5|true]!) @ {}\\~~~~~~ pred [D|true] \land P)]
\end{zedrule}

The first type proviso in the SchemaExists rule checks that
declaration lists $D1$ and~$D2$ are type compatible.  The
$\schemaminus$ proviso subtracts the normalized declaration list
of~$D1$ from the normalized declaration list of~$D2$.  The rules for
schema $\exists_1$ and $\forall$ are similar.


\begin{zedrule}{SchemaExists}
   ([D1 | true] \land [D2 | true]) :: \power [D3 | true] \\
   [D1|true] :: \power [D4|true] \\
   [D2|true] :: \power [D5|true] \\
   ([D5|true] \schemaminus [D4|true]) \is [D6|true]
\derives
   (\exists D1 | P1 @ [D2|P2]) = {}\\~~ [D6 | (\exists D1 @ P1 \land pred [D2|true] \land P2)]
\end{zedrule}

\begin{zedrule}{SchemaExists1}
   ([D1 | true] \land [D2 | true]) :: \power [D3 | true] \\
   [D1|true] :: \power [D4|true] \\
   [D2|true] :: \power [D5|true] \\
   ([D5|true] \schemaminus [D4|true]) \is [D6|true]
\derives
   (\exists_1 D1 | P1 @ [D2|P2]) = {}\\~~ [D6 | (\exists_1 D1 @ P1 \land pred [D2|true] \land P2)]
\end{zedrule}


\begin{zedrule}{SchemaForall}
   ([D1 | true] \land [D2 | true]) :: \power [D3 | true] \\
   [D1|true] :: \power [D4|true] \\
   [D2|true] :: \power [D5|true] \\
   ([D5|true] \schemaminus [D4|true]) \is [D6|true]
\derives
   (\forall D1 | P1 @ [D2|P2]) = {}\\~~ [D6 | (\forall D1 @ P1 \land pred [D2|true] \land P2)]
\end{zedrule}

\begin{zedrule}{SchemaHiding}
  E ::\power [D1| true] \\
  [D1|true] \hide (NL) \is {}\\~~ \exists [D2|true] @ [D1|true]
\derives
  E \hide (NL) = \exists [D2|true] @ E
\end{zedrule}


\begin{zedrule}{GenInst}
  N [EL] :: \power [D | true] \\
  N [EL] \hasDefn E2
\derives
  N [EL] = E2
\end{zedrule}

The Ref rule uses the lookup proviso to expand schema names by its
definition.

\begin{zedrule}{Ref}
  N :: \power [D | true] \\
  N \hasDefn E2
\derives
  N = E2
\end{zedrule}

This rule expands any remaining $\Delta$ schemas.  If the
specification explicitly defined the $\Delta$ schema, then the above
Ref rule would have replaced it.  There is a similar rule for $\Xi$
schema names.

\begin{zedrule}{DeltaRef}
  \Delta \unprefix N \is N2 \\
\derives
  N = [N2; N2~']
\end{zedrule}

\begin{zedrule}{XiRef}
  \Xi \unprefix N \is N2 \\
  [N2|true] :: \power [D2 | true] \\
  \theta [D2|true] = \theta [D2|true]~' \iff P
\derives
  N = [N2; N2~' | P]
\end{zedrule}

These are recently-added (Mar 2008) rules to unfold schemas that
are written as set comprehensions or set enumerations etc.
We need a more general version of these, but it is difficult to find 
a more general form that does not produce infinite recursive unfolding!

\TBDpar{IT}{The aim of the rules in section 5.2 is to unfold - by rewriting -
any schema expression (one whose type is set of bindings)
towards a schema construction expression.
These new rules here do rewrite to schema construction expressions,
but going from set extensions of binding expressions seems
more like folding than unfolding.
Cadiz never needed such inference steps,
and so I wonder why are they needed by CZT?}

%\begin{zedrule}{SchemaSet1}
%  \{E1\} :: \power [D | true]
%\derives
%  \{E1\} = [D | \theta [D|true] = E1 ]
%\end{zedrule}
\begin{zedrule}{SchemaSet2}
  \{E1,E2\} :: \power [D | true]
\derives
  \{E1,E2\} = {}\\~~ [D | \theta [D|true] = E1 \lor \theta [D|true] = E2 ]
\end{zedrule}
\begin{zedrule}{SchemaSet3}
  \{E1,E2,E3\} :: \power [D | true]
\derives
  \{E1,E2,E3\} = {}\\~~ [D | {}\\~~~~ \theta [D|true] = E1 \lor {}\\~~~~ \theta [D|true] = E2 \lor {}\\~~~~ \theta [D|true] = E3 ]
\end{zedrule}
\begin{zedrule}{SchemaSet4}
  \{E1,E2,E3,E4\} :: \power [D | true]
\derives
  \{E1,E2,E3,E4\} = {}\\~~ [D | {}\\~~~~ \theta [D|true] = E1 \lor {}\\~~~~ \theta [D|true] = E2 \lor {}\\~~~~ \theta [D|true] = E3 \lor {}\\~~~~ \theta [D|true] = E4 ]
\end{zedrule}


Predicates:

\begin{zedrule}{SchemaPred}
  [D|P] \iff (pred [D|true] \land P)
\end{zedrule}
