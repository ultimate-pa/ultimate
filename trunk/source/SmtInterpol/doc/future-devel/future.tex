\documentclass{article}
\usepackage[english]{babel}
\usepackage{xspace}
\usepackage{hyperref}

\newcommand\SI{SMTInterpol\xspace}
\newcommand\version{2.0pre\xspace}

\title{\SI\ - Development Plan}

\author{J\"urgen Christ, Jochen Hoenicke\\
  University of Freiburg\\
  \texttt{\{christj,hoenicke\}@informatik.uni-freiburg.de}}

\begin{document}
\maketitle
\section{Current Version}
This document covers the version \SI \version that will participate in the
SMT-COMP 2011.  See the system description for further details.

\section{General Development Plan}
The future development of \SI should be guided by the requirements of the
model checking tools that use \SI as backend.  These tools currently include
Trace Abstraction, Slicing Abstraction, and Lazy Abstraction with the
corresponding requirement providers Matthias Heizmann, Evren Ermis, and
Alexander Nutz.

\section{Detailed Development Plan}
This section lists the detailed development steps for \SI.  The requirements
and the providers are listed in the description.

\subsection{Competition Cleanup}
After the SMT-COMP 2011, all unsoundness results and bugs revealed during the
competition should be fixed.  This step is a general requirement for all
further development steps.

\subsection{Interpolation}
Interpolants shall be derived from the proof trees stored inside the solver.
All users have this requirement.  Currently, we have an interpolator for the
propositional core and the congruence closure.  An interpolator for linear
arithmetic, and the theory combination are still to be done.

\subsection{Models}
The current solver produces and dumps a model in every theory solver.  These
partial models should be joined to form a complete model.  Additionally, the
SMTLIB~2 model query features should be implemented on top of this model,
i.e., it should be possible to ask for values of specific terms in a model.

This requirement currently comes from Evren Ermis to trace back the path by
detecting the parts of a disjunction that evaluate to true.  This information
is extracted using block labels that may be implemented using SMTLIB~2 named
terms.  Extraction of this information can be done using
\textbf{get-assignment} for named terms or \textbf{get-value} for arbitrary
quantifier-free closed terms.

\subsection{Arrays and Quantifiers}
To express richer properties, we need support for arrays and quantifiers in
the solver and the interpolator.  Although theoretically developed, a concrete
and well performing implementation is still missing.  The remaining question
here is how to support arrays.  Although we could support arrays through
e-matching, specialized array decision procedures seem to have a better
performance.

The interpolation part has to be changed to support quantifier inference.
Furthermore, model production becomes more difficult and - in general -
impossible.

Furthermore, a quantifier model checker based on MBQI could be implemented on
top of the basic solver routine.  This checker needs some configuration of the
number of iterations, i.e., checks it is allowed to perform to prevent
non-termination.  Including this technique, we get a sound and complete solver
for some fragments of quantified theories.

The requirement comes from all providers and automatic translation tools used
to generate Boogie input from C programs.

\subsection{Non-linear Arithmetic}
During verification of software, a small number of constraints asserted in the
SMT solver end up as linear constraints.  For these so called ``large largely
linear'' systems, the computation of Gr\"obner bases proved very efficient.
Based on this approach, a non-linear arithmetic solver could be added to \SI.
The interpolation of non-linear constraints might be done by a saturation
based Gr\"obner bases construction loop and a separating term ordering.
Soundness of this method is still to be determined.

This requirement appears in a benchmark from Alexander Nutz and in some VCC
benchmarks run by Matthias Heizmann.

\end{document}
