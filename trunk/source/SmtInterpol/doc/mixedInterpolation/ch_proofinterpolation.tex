\section{Proof Tree-Based Interpolation}

Interpolants can be computed from proofs of unsatisfiability as
Pudl\'ak and McMillan have already shown.  In this section we will
introduce their algorithms.  Then, we will discuss the changes
necessary to handle mixed literals introduced, e.\,g., by theory combination.

\subsection{Pudl\'ak's and McMillan's Interpolation Algorithms}

Pudl\'ak's and McMillan's algorithms assume that the pivot literals are not mixed.
We will remove this restriction later.  We define a common
framework that is more general and can be instantiated to obtain Pudl\'ak's or McMillan's
algorithm to compute interpolants.  For this, we use two projection
functions on literals $\cdot\proj A$ and $\cdot\proj B$ as defined
below.
They have the properties (i)
$\symb(\ell\proj A)\subseteq\symb(A)$, (ii) $\symb(\ell\proj
B)\subseteq\symb(B)$, and (iii) $\ell \iff (\ell\proj A \land
\ell\proj B)$.  Other projection functions are possible and this
allows for varying the strength of the resulting interpolant as shown
in \cite{D'Silva2010}.  We extend the projection function to
conjunctions of literals component-wise.

\medskip
\centerline{
  \renewcommand{\arraystretch}{1.1}
 \begin{tabular}{l|c|c|c|c|}
    & \multicolumn{2}{c|}{Pudl\'ak}
    & \multicolumn{2}{c|}{McMillan}\\
    &$\ell\proj A$&$\ell\proj B$
    &$\ell\proj A$&$\ell\proj B$\\
   \hline
    $\ell$ is $A$-local
    & $\ell$ & $\top$ & $\ell$ & $\top$ \\
    $\ell$ is $B$-local
    & $\top$ & $\ell$ & $\top$ & $\ell$ \\
    $\ell$ is shared
    & $\ell$ & $\ell$ & $\top$ & $\ell$
  \end{tabular}
}
\medskip

Given an interpolation problem $A$ and $B$, a \emph{partial
interpolant} of a clause $C$ is an interpolant of the formulae $A
\land (\lnot C \proj A)$ and $B \land (\lnot C \proj B)$\footnote{Note that $\lnot C$ is a conjunction of literals. Thus, 
$\lnot C\proj A$ is well defined.}.
%
Partial interpolants can be computed inductively over the structure of
the proof tree.  A partial interpolant of a theory lemma $C$ can be
computed by a theory-specific interpolation routine as an interpolant
of $\lnot C \proj A$ and $\lnot C \proj B$. Note that the conjunction
is equivalent to $\lnot C$ and therefore unsatisfiable.  For an input
clause $C$ from the formula $A$ (resp.\ $B$), a partial interpolant is
$\lnot(\lnot C\setminus A)$ (resp.\ $\lnot C \setminus B$) where
$\lnot C\setminus A$ is the conjunction of all literals of $\lnot C$ 
that are not in $\lnot C \proj A$ and analogously for $\lnot C\setminus B$.  For a
resolution step, a partial interpolant can be computed using (\ref{rule:res}),
which is given below. For this rule, it is easy to show that $I_3$ is a partial
interpolant of $C_1\lor C_2$ given that $I_1$ and $I_2$ are partial
interpolants of $C_1\lor \ell$ and $C_2\lor \lnot \ell$, respectively. Note
that the ``otherwise'' case never triggers in McMillan's algorithm.
%
\begin{equation}\tag{rule-res}\label{rule:res}
\inferrule{C_1\lor \ell : I_1 \quad C_2\lor \lnot \ell : I_2}
            {C_1\lor C_2 : I_3} \quad 
	    \text{where }I_3 = \begin{cases}
	      I_1 \lor I_2 & \text{if }\ell\proj B = \top\\
	      I_1 \land I_2 & \text{if }\ell\proj A = \top\\
	      \begin{array}{l}(I_1\lor\ell) \land{}\\
	      (I_2\lor \lnot \ell)\end{array} & \text{otherwise}
	    \end{cases}
\end{equation}
%
As the partial interpolant of the root of the proof tree (which is
labelled with the clause $\bot$) is an interpolant of the input
formulae $A$ and $B$, this algorithm can be used to compute interpolants.

\begin{theorem}
 The above-given partial interpolants are correct, i.e., if
  $I_1$ is a partial interpolant of $C_1 \lor \ell$ 
  and $I_2$ is a partial interpolant of $C_2 \lor \lnot \ell$ 
  then $I_3$ is a partial interpolant of the  clause $C_1 \vee C_2$.
\end{theorem}

\begin{techreport}
\begin{proof}
 The third property, i.e., $\symb(I_3) \subseteq \symb(A) \cap \symb(B)$, 
 clearly holds
 if we assume it holds for $I_1$ and $I_2$. Note that in the
 ``otherwise'' case, $\ell$ is shared.
%
 We prove the other two partial interpolant properties separately.
 \subsubsection*{Inductivity.} We have to show
 \[A \land \lnot C_1\proj A \land \lnot C_2\proj A \models I_3.\]
 For this we use the inductivity of $I_1$
 and $I_2$:
 \begin{align*}
  & A \land \lnot C_1 \proj A \land \lnot \ell\proj A \models I_1 \tag{ind1} \\
  & A \land \lnot C_2 \proj A \land \ell\proj A \models I_2 \tag{ind2}
 \end{align*}  

 Assume $A$, $\lnot C_1\proj A$, and $\lnot C_2\proj A$. Then, (ind1)
 simplifies to $\lnot\ell \proj
 A \rightarrow I_1$ and (ind2) simplifies to $\ell \proj A \rightarrow I_2$.  We show that
 $I_3$ holds under these assumptions.

\paragraph{Case $\ell\proj B = \top$.}
 
 Then by the definition of the projection function, $\ell\proj A = \ell$ 
 and $\lnot \ell\proj A = \lnot \ell$ hold. If $\ell$ holds, (ind2) gives us
 $I_2$, otherwise (ind1) gives us $I_1$, thus $I_3 = I_1\lor I_2$ holds in both cases.
 
\paragraph{Case $\ell\proj A = \top$.}
 
 Then (ind1) gives us $I_1$ because $\lnot \ell\proj A = \top$ 
 (the negation of $\ell$ is still not in $A$), and (ind2) gives us $I_2$. 
 So $I_3 = I_1 \land I_2$ holds.
 
\paragraph{Case ``otherwise''.}
 
 By the definition of the projection function
 $\ell\proj A = \ell\proj B = \ell$ and 
 $\lnot \ell\proj A = \lnot \ell\proj B = \lnot \ell$. If $\ell$ holds,
 the left conjunct $(I_1 \lor \ell)$ of $I_3$ holds and the right
 conjunct $(I_2\lor \lnot\ell)$ of $I_3$ is fulfilled because (ind2) gives us $I_2$.
 If $\lnot \ell$ holds, (ind1) gives us $I_1$ and both conjuncts of $I_3$ hold.
 
\subsubsection*{Contradiction.} 
 We have to show:
 
 \[B \land \lnot C_1\proj B \land \lnot C_2\proj B \land I_3 \models \bot\]
 
 We use the contradiction properties of $I_1$ and $I_2$:
 \begin{align*}
  & B \land \lnot C_1\proj B \land \lnot \ell\proj B \land I_1 \models \bot \tag{cont1}\\
  & B \land \lnot C_2\proj B \land \ell \proj B \land I_2 \models \bot \tag{cont2}
 \end{align*}  

 If we assume $B$, $\lnot C_1\proj B$, and $\lnot C_2\proj B$, (cont1)
 simplifies to $\lnot \ell\proj B \land I_1 \rightarrow \bot$ and (cont2)
 simplifies to $\ell\proj B \land I_2 \rightarrow \bot$. We show $I_3
 \rightarrow \bot$.

 \paragraph{Case $\ell\proj B = \top$.}

 Then (cont1) and $\lnot \ell\proj B = \top$ give us $I_1 \rightarrow \bot$,
 and (cont2) and $\ell \proj B = \top$ give us $I_2 \rightarrow \bot$. 
 Thus $I_3 \equiv I_1 \lor I_2$ is contradictory.

 \paragraph{Case $\ell\proj A = \top$.}

 Then $\ell \proj B = \ell$ and $\lnot \ell \proj B = \lnot \ell$. Then, if 
 $\ell$ holds, (cont2) gives us $I_2 \rightarrow \bot$. If $\lnot \ell$ holds, 
(cont1) gives us
 $I_1 \rightarrow \bot$ analogously. In both cases,
 $I_3 \equiv I_1 \land I_2$ is contradictory.

 \paragraph{Case ``otherwise''.}
 
 By the definition of the projection function 
 $\ell\proj A = \ell\proj B = \ell$ and 
 $\lnot \ell\proj A = \lnot \ell\proj B = \lnot \ell$ hold.
 Assuming $I_3 \equiv (I_1\lor\ell) \land (I_2\lor \lnot\ell)$ holds, we prove
 a contradiction.
 If $\ell$ holds, the second conjunct of $I_3$ implies $I_2$.
 Then, (cont2) gives us a contradiction. 
 If $\lnot \ell$ holds, the first conjunct of $I_3$ implies $I_1$
 and (cont1) gives us a contradiction.
%, 
\qed 
\end{proof}
\end{techreport}

