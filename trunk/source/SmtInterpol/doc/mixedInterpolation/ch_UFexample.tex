\subsection{Example}

We demonstrate our algorithm on the following example:
\begin{align*}
  A\equiv&(\lnot q \lor a=s_1)\land(q \lor a=s_2)\land f(a)=t\\
  B\equiv&(\lnot q \lor b=s_1)\land(q \lor b=s_2)\land f(b)\neq t
\end{align*}
The conjunction $A\land B$ is unsatisfiable.  In this example, $a$ is
$A$-local, $b$ is $B$-local and the remaining symbols are shared.

Assume the theory solver for \euf introduces the mixed literal $a=b$ and
provides the lemmas (i) $f(a)\neq t\lor a\neq b\lor 
f(b)=t$, (ii) $a\neq s_1\lor b\neq s_1\lor a=b$, and (iii) $a\neq
s_2\lor b\neq s_2\lor a=b$.  
Let the variable $x$ be
associated with the equality $a=b$.  Then, we label the lemmas with (i)
$f(x)=t$, (ii) $EQ(x,s_1)$, and (iii) $EQ(x,s_2)$.

We compute an interpolant for $A$ and $B$ using Pudl\'ak's algorithm.  Since
the input is already in conjunctive normal form, we can directly apply resolution.
\begin{tacas}
  From lemma (ii) and the input clauses $\lnot q \lor a=s_1$ and $\lnot q \lor
  b=s_1$ we can derive the clause $\lnot q\lor a=b$.  The partial interpolant
  of the derived clause is still $EQ(x,s_1)$, since the partial interpolants
  of the input clauses are $\bot$ resp.\ $\top$.  Similarly, from 
  lemma (iii) and the input clauses $q \lor a=s_2$
  and $q \lor b=s_2$ we can derive the clause $q\lor a=b$ with partial
  interpolant $EQ(x,s_2)$.  A resolution step on these two clauses with $q$ as
  pivot yields the clause $a=b$. Since $q$ is a shared literal, Pudl\'ak's
  algorithm introduces 
  the case distinction.  Hence, we get the partial interpolant
  $(EQ(x,s_2)\lor q)\land(EQ(x,s_1)\lor \lnot q)$.  Note that this interpolant
  has the
  form $I_1[EQ(x,s_1)][EQ(x,s_2)]$ and, therefore, satisfies the syntactical
  restrictions required by our scheme.

  From the \euf-lemma (i) and the input clauses $f(a)=t$ and $f(b)\neq t$, we
  can derive the clause $a\neq b$ with partial interpolant $f(x)=t$.  Note that
  this interpolant has the form $I_2(x)$ which also corresponds to the syntactical
  restrictions needed for our method.

  If we apply the final resolution step on the mixed literal $a=b$ using
  (\ref{rule:inteq}), we get the
  interpolant $I_1[I_2(s_1)][I_2(s_2)]$ which corresponds to the interpolant
  $(f(s_2)=t\lor q)\land(f(s_1)=t\lor \lnot q)$.
\end{tacas}
\begin{techreport}
  Note that for Pudl\'ak's algorithm every input clause has the partial
  interpolant $\bot$ ($\top$) if it is part of $A$ ($B$).  In the following
  derivation trees we apply the following simplifications without explicitely
  stating them:
  \begin{align*}
    F\land\top&\equiv F\\
    F\lor\bot&\equiv F
  \end{align*}

  From lemma (ii) and the input clauses $\lnot q \lor a=s_1$ and $\lnot q \lor
  b=s_1$ we can derive the clause $\lnot q\lor a=b$.  The partial interpolant
  of the derived clause is still $EQ(x,s_1)$.
  \[
  \inferrule*{ \inferrule*{
    \lnot q \lor a=s_1 : \bot \\
    a\neq s_1\lor b\neq s_1\lor a=b : EQ(x,s_1)}
            {b\neq s_1\lor \lnot q \lor a=b : EQ(x,s_1)}\\
            b=s_1\lor\lnot q : \top}
            {\lnot q\lor a=b : EQ(x,s_1)}
  \]
  Similarly, from 
  lemma (iii) and the input clauses $q \lor a=s_2$
  and $q \lor b=s_2$ we can derive the clause $q\lor a=b$ with partial
  interpolant $EQ(x,s_2)$.
  \[
  \inferrule*{ \inferrule*{
    q \lor a=s_2 : \bot \\
    a\neq s_2\lor b\neq s_2\lor a=b : EQ(x,s_2)}
            {b\neq s_2\lor q \lor a=b : EQ(x,s_2)}\\
            b=s_2\lor q : \top}
            {q\lor a=b : EQ(x,s_2)}
  \]
  A resolution step on these two clauses with $q$ as
  pivot yields the clause $a=b$. Since $q$ is a shared literal, Pudl\'ak's
  algorithm introduces 
  the case distinction.  Hence, we get the partial interpolant
  $(EQ(x,s_2) \lor q)\land(EQ(x,s_1) \lor \lnot q)$.  Note that this interpolant
  has the
  form $I_1[EQ(x,s_1)][EQ(x,s_2)]$ and, therefore, satisfies the syntactical
  restrictions.
  \[
  \inferrule*
      { q\lor a=b : EQ(x,s_2) \\ \lnot q\lor a=b : EQ(x,s_1) }
      {a=b : (EQ(x,s_2) \lor q)\land(EQ(x,s_1) \lor \lnot q)}
  \]
  From the \euf-lemma (i) and the input clauses $f(a)=t$ and $f(b)\neq t$, we
  can derive the clause $a\neq b$ with partial interpolant $f(x)=t$.  Note that
  this interpolant has the form $I_2(x)$ which also corresponds to the syntactical
  restrictions needed for our method.
  \[
  \inferrule*
  {
  \inferrule*
  { f(a)=t : \bot \\ f(a)\neq t\lor a\neq b\lor f(b)=t : f(x)=t }
  { f(b)=t \lor a\neq b : f(x)=t } \\ f(b)\neq t : \top }
  { a\neq b : f(x)=t }
  \]
  If we apply the final resolution step on the mixed literal $a=b$ using
  (\ref{rule:inteq}), we get the
  interpolant $I_1[I_2(s_1)][I_2(s_2)]$ which corresponds to the interpolant
  $(f(s_2)=t\lor q)\land(f(s_1)=t \lor \lnot q)$.
  \[
  \inferrule*
  { a=b : (EQ(x,s_2) \lor q)\land(EQ(x,s_1) \lor \lnot q) \\ a\neq b : f(x)=t }
  { \bot : (f(s_2)=t \lor q)\land(f(s_1)=t \lor \lnot q) }
  \]
  When resolving on $q$ in the derivations above, the mixed literal $a=b$
  occurs in both antecedents.  This leads to the form
  $I[EQ(x,s_1)][EQ(x,s_2)]$.  We can prevent this by resolving in a different
  order.
  We could first resolve the clause $q\lor a=b$ with the clause $a\neq b$ and
  obtain the partial interpolant $f(s_2)=t$ using (\ref{rule:inteq}).
  \[
  \inferrule*{ a=b \lor q : EQ(x,s_2) \\ a\neq b : f(x) = t }
            { q : f(s_2) = t }
  \]
  Then we could resolve the clause $\lnot q\lor a=b$ with the clause $a\neq b$ and
  obtain the partial interpolant $f(s_1)=t$ again using (\ref{rule:inteq}).
  \[
  \inferrule*{ a=b \lor \lnot q : EQ(x,s_1) \\ a\neq b : f(x) = t }
    { \lnot q : f(s_1) = t }
  \]
  The final resolution step on $q$ will then introduce the case distinction
  according to Pudl\'ak's algorithm.  This results in the same interpolant.
  \[
  \inferrule*
  { q : f(s_2)=t \\
    \lnot q : f(s_1)=t }
  { \bot : (f(s_2)=t \lor q)\land(f(s_1)=t \lor \lnot q) }
  \]
\end{techreport}
