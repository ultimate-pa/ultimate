\begin{abstract}
%  Craig interpolants are widely used in model checking and state space
%  abstraction. Interpolants typically are extracted from proofs produced by
%  theorem provers. While this extraction procedure is easy and well understood
%  in the context of propositional logic, extracting interpolants from a proof
%  generated by an SMT solver is more complex. In contrast to SAT solvers, SMT
%  solvers create new literals, e.g., to combine multiple theories in a
%  Nelson-Oppen style or to split the solution space using cuts. These literals
%  might contain symbols local to different parts of the interpolation
%  problem. Such literals are called \emph{mixed}, or, sometimes,
%  \emph{uncolourable}. Resolution steps on mixed literals are the major
%  difficulty when extracting interpolants from proofs from SMT
%  solvers.

  Craig interpolation in SMT is difficult because, e.\,g., theory combination
  and integer cuts introduce mixed literals, i.\,e., literals
  containing local symbols from both input formulae.  In this paper,
  we present a scheme to compute Craig interpolants in the presence of
  mixed literals.  Contrary to existing approaches, this scheme
  neither limits the inferences done by the SMT solver, nor does it
  transform the proof tree before extracting interpolants.  Our scheme
  works for the combination of uninterpreted functions and linear
  arithmetic but is extendable to other theories.  The
  scheme is implemented in the interpolating SMT solver SMTInterpol.
\end{abstract}
