%% LyX 2.0.3 created this file.  For more info, see http://www.lyx.org/.
%% Do not edit unless you really know what you are doing.
\documentclass[english]{scrartcl}
\usepackage[T1]{fontenc}
\usepackage[latin9]{inputenc}
\setlength{\parskip}{\smallskipamount}
\setlength{\parindent}{0pt}
\usepackage{units}
\usepackage{amsthm}
\usepackage{amsmath}
\usepackage{amssymb}

\makeatletter
%%%%%%%%%%%%%%%%%%%%%%%%%%%%%% User specified LaTeX commands.
\usepackage{mathpartir}

\makeatother

\usepackage{babel}
\begin{document}

\title{Interpolation of Clauses Containing Mixed Literals in the Theories
of Linear Arithmetic and Uninterpreted Functions}

\maketitle

\subsubsection*{Problem Description:}

We want to compute the Craig-interpolant for two formulas $A$ and
$B$ from the resolution tree (/DAG) of the unsatisfiability proof
for $A\wedge B$ in a given Theory $\mathcal{T}$. This is done by
computing partial interpolants for every node in the tree. The partial
interpolant of a leaf-node depends on which of its literals are from
$A$ and which are from $B$, the interpolant is computed depending
on this partition of the literals. For input clauses the partition
is clear, but theory lemmas may have so-called mixed literals, that
is literals that contain symbols which are $A$-local as well as symbols
which are $B$-local, so they cannot be assigned to one part of the
partition. These cases are treated in the following for the quantifier-free
theories of linear arithmetic and equality with uninterpreted functions.


\subsubsection*{Definitions:}

A variable $x$ is considered to be \emph{$A$-local}, if $x\in syms\left(A\right)\backslash syms\left(B\right)$,
\emph{$B$-local}, if $x\in syms\left(B\right)\backslash syms\left(A\right)$
and \emph{shared} if $x\in syms\left(A\right)\cap syms\left(B\right)$.

A \emph{partial interpolant} $I$ of $A$ and $B$ at clause $C$
is a formula such that $A\wedge\neg C\restriction A\rightarrow I$
and $B\wedge\neg C\restriction B\wedge I\rightarrow\bot$. Where $X\restriction F$
means the restriction of the formula $X$ to symbols from $F$. This
may be achieved in various ways. In the following, given only two
formulas $A$ and $B$, we will use $A\backslash B$ for $A\restriction A$
and $B\downarrow B$ for $B\restriction B$. Where $A\backslash B$
is $A$ modified such that it only contains symbols from $B$ ($B$-local
or shared), and $B\downarrow B$ is $B$ with only literals that contain
symbols which are not $B$-local.


\section{Mixed Equalities}


\subsubsection*{}

Theory lemmas may contain literals of the form $a=b$ or $a\neq b$
with $a$ being $A$-local and $b$ being $B$-local. The first approach
to solve this is by introducing a fresh variable $x$ which is considered
to be shared between $A$ and $B$. So we represent $a=b$ equivalently
by 

\[
a=b\leftrightarrow\exists x.a=x\wedge x=b
\]


which allows partitioning of the literals.


\subsubsection*{Definitions:}

\begin{eqnarray*}
\left(a=b\right)\backslash B & :\equiv & a=x\\
\left(a=b\right)\downarrow B & :\equiv & x=b\\
\left(a\neq b\right)\backslash B & :\equiv & a=x\\
\left(a\neq b\right)\downarrow B & :\equiv & x\neq b
\end{eqnarray*}



\subsubsection*{Claim:}

\begin{eqnarray}
a=b & \leftrightarrow & \exists x.\left(a=b\right)\backslash B\wedge\left(a=b\right)\downarrow B\\
a\neq b & \leftrightarrow & \exists x.\left(a\neq b\right)\backslash B\wedge\left(a\neq b\right)\downarrow B
\end{eqnarray}



\subsubsection*{Proofs of equivalence:}

(1): $\rightarrow$: choose $x:=a$; $\leftarrow$: by transitivity

(2): $\rightarrow$: choose $x:=a$, then $x\neq b$; $\leftarrow$:
From $\left(a\neq b\right)\backslash B$, we get $a=x$. Then if $a=b$,
then $x=b$ by transitivity and symmetry, contradiction to assumption.

~

So, by computing interpolants for the modified formulas, we get interpolants
for the original formula as well. Thus given a formula $F=F_{A}\wedge F_{B}\wedge a=b$
with $F_{A}$ and $F_{B}$ being the $A$-local part and the part
with only variables from $B$ respectively, we get $F\leftrightarrow F\restriction A\wedge F\restriction B$
with $F\restriction A=F_{A}\wedge\left(a=b\right)\backslash B$ and
$F\restriction B=F_{B}\wedge\left(a=b\right)\downarrow B$. 

~

This solution has the problem that it introduces a new quantifier
into each formula which is typically expensive to eliminate later.
But as the new variable is defined in a certain way, we can eliminate
the quantifier further ahead in the resolution process.

As we have a proof of unsatisfiability, every literal has to be eliminated
until the end in order to achieve the empty clause. We can eliminate
the quantifier and the new variable when $a=b$ is pivoted using the
following interpolation rule:

\[\inferrule {(a=b) \vee C_1:I_1[x=s] \\ a \neq b \vee C_2: I_2(x) } {C_1 \vee C_2: I_1[I_2(s)] } \]

Note: it can rightfully be assumed that $I_{1}$ depends on $x=s$
and that $I_{2}$ depends on $s$ because of some properties congruence
closure algorithm which is used for computing the Interpolant of the
respective $F\restriction A\wedge F\restriction B$.

To proof the correctness of this rule we have to show that $I_{1}[I_{2}(s)]$
is indeed a partial interpolant of $A$ and $B$ for the clause $C_{1}\vee C_{2}$:

\begin{eqnarray}
A\wedge\neg C_{1}\backslash B\wedge\neg C_{2}\backslash B & \models & I_{1}[I_{2}(s)]\label{eq:1}\\
B\wedge\neg C_{1}\downarrow B\wedge\neg C_{2}\downarrow B\wedge I_{1}[I_{2}(s)] & \models & \bot\label{eq:2}
\end{eqnarray}



\subsubsection*{Claim:}

First, we convince ourselves that a partial interpolant $I[\psi]$
is monotonic wrt. the substitution of the formula $\psi$ that only
occurs positive. This means that for any two given formulas $\phi_{1},\phi_{2}$:

\begin{eqnarray*}
\phi_{1}\rightarrow\phi2 & \models & I[\phi_{1}]\rightarrow I[\phi_{2}]
\end{eqnarray*}



\subsubsection*{Proof:}

by belief...


\subsubsection*{Proof of (\ref{eq:1}):}

We know that $I_{1}$ and $I_{2}$ are partial interpolants. The following
formulas follow from their inductivity property: 

\begin{eqnarray}
A\wedge\neg C_{1}\backslash B\wedge a=x & \models & I_{1}[x=s]\label{eq:3}\\
A\wedge\neg C_{2}\backslash B\wedge a=x & \models & I_{2}(x)\label{eq:4}
\end{eqnarray}


Assume $A$, $\neg C_{1}\backslash B$, $\neg C_{2}\backslash B$.
(Then the validity of $I_{1}[I_{2}(s)]$ remains to be shown.)

By (\ref{eq:3}) we know $a=x\models I_{1}\left[x=s\right]$, \\
thus it follows that $\models I_{1}\left[a=s\right]$.({*})

By (\ref{eq:4}) we know that $\models a=x\rightarrow I_{2}\left(x\right)$,
thus also $\models a=s\rightarrow I_{2}\left(s\right)$. \\
By monotonicity of $I_{1}[x=s]$, we get $\models I_{1}\left[a=s\right]\rightarrow I_{1}\left[I_{2}\left(s\right)\right]$.
\\
Together with ({*}) we are done.


\subsubsection*{Proof of (\ref{eq:2}):}

From the contradictory property of the partial interpolants $I_{1}$
and $I_{2}$ it follows that:

\begin{eqnarray}
B\wedge\neg C_{1}\downarrow B\wedge x=b\wedge I_{1}\left[x=s\right] & \models & \bot\label{eq:5}\\
B\wedge\neg C_{2}\downarrow B\wedge x\neq b\wedge I_{2}\left(x\right) & \models & \bot\label{eq:6}
\end{eqnarray}


Assume $B,\neg C_{1}\downarrow B,\neg C_{2}\downarrow B$. Then $I_{1}\left[I_{2}\left(s\right)\right]\models\bot$
remains to be shown.

By (\ref{eq:5}) we know that $x=b\wedge I_{1}\left[x=s\right]\models\bot$.
\\
It follows that $I_{1}\left[b=s\right]\models\bot$. ({*})

By (\ref{eq:6}) we know that $\models x\neq b\wedge I_{2}\left(x\right)\rightarrow\bot$.
\\
By contraposition it follows that $\models I_{2}\left(x\right)\rightarrow x=b$,
and also $\models I_{2}\left(s\right)\rightarrow s=b$. \\
By monotonicity we get $\models I_{1}\left[I_{2}\left(s\right)\right]\rightarrow I_{1}\left[s=b\right]$.
\\
Together with ({*}) we are done.


\section{Mixed Rational Inequalities}

Theory lemmas may also contain literals of the form $t\leq0$ where
$t$ contains symbols local to $A$ and $B$. So, again, we cannot
partition the corresponding Formula. 

Let $t=t_{A}-t_{B}$ with $t_{A}/t_{B}$ being defined as $A/B$-local.

We introduce a fresh auxiliary variable, named $y$. And we define

\begin{eqnarray*}
\left(t\leq0\right)\backslash B & :\equiv & t_{A}-y\leq0\\
\left(t\leq0\right)\downarrow B & :\equiv & y-t_{B}\leq0\\
\left(t>0\right)\backslash B & :\equiv & t_{A}-y\geq0\\
\left(t>0\right)\downarrow B & :\equiv & y-t_{B}>0
\end{eqnarray*}



\subsubsection*{Claim:}

\begin{eqnarray}
t\leq0 & \leftrightarrow & \exists y.\left(t_{A}-y\leq0\wedge y-t_{B}\leq0\right)\label{eq:7}\\
t>0 & \leftrightarrow & \exists y.\left(t_{A}-y\geq0\wedge y-t_{B}>0\right)\label{eq:8}
\end{eqnarray}



\subsubsection*{Proof:{[}..{]}}

(\ref{eq:7}): $\rightarrow$: clearly $y:=t_{A}$ satisfies both
conjuncts (given our assumption $t=t_{A}-t_{B}$). $\leftarrow$:
clear by adding up both sides of the equations.

(\ref{eq:8}): $\rightarrow$: clearly $y:=t_{A}$ satisfies both
conjuncts. $\leftarrow$: clear by adding up both sides of the equations.

~

Interpolation for the modified clauses of the leafs of the resolution
tree is done by adding up the $A$-partition of each clause. This
yields a $\leq/>$-clause depending on $y$.


\subsubsection*{Definition:}

\begin{eqnarray*}
EQ_{R}\left(s,F\right) & :\equiv & s\leq0\wedge\left(s<0\vee F\right)
\end{eqnarray*}


This formula will be used to express the partial interpolant at the
point, when $t\leq0$ is pivoted and we want to get rid of the auxiliary
variable ($y$). 

Note that $EQ_{R}\left(s,\top\right)\equiv s\leq0$ and $EQ_{R}\left(s,\bot\right)\equiv s<0$.
Hence each interpolant of the leaves can be expressed this way, too.
{[}braucht's noch mehr??{]}

The pivoting rule for $t\leq0$:

\[\inferrule {t\leq 0 \vee C_1 : I_1[EQ_R (c_1y + s_1, F_1(y)] \\ t > 0 \vee C_2: I_2[EQ_R (c_2y - s_2, F(y)} {C_1 \vee C_2: I_1[I_2[EQ_R(c_2 s_1 + c_1 s_2, F_1 (\frac{s_{2}}{c_{2}}) \wedge F_2(\frac{s_{2}}{c_{2}})] } \]

$c_{1},c_{2}$ are positive constants, $s_{1},s_{2}$ are terms over
shared symbols.

In the remainder, of this section let's call 
\begin{eqnarray*}
EQ_{R}^{\left(1\right)}\left(y\right) & :\equiv & EQ_{R}\left(c_{1}y+s_{1},F_{1}\left(y\right)\right)\\
EQ_{R}^{\left(2\right)}\left(y\right) & :\equiv & EQ_{R}\left(-c_{2}y+s_{2},F_{2}\left(y\right)\right)\\
EQ_{R}^{\left(3\right)} & :\equiv & EQ_{R}\left(c_{2}s_{1}+c_{1}s_{2},F_{1}\left(\frac{s_{2}}{c_{2}}\right)\wedge F_{2}\left(\frac{s_{2}}{c_{2}}\right)\right)\\
I_{l} & :\equiv & I_{1}\left[EQ_{R}^{\left(1\right)}\left(y\right)\right]\\
I_{r} & :\equiv & I_{2}\left[EQ_{R}^{\left(2\right)}\left(y\right)\right]\\
I & :\equiv & I_{1}\left[I_{2}\left[EQ_{R}^{\left(3\right)}\right]\right]
\end{eqnarray*}



\subsubsection*{Claim:}

$I$ is a partial interpolant of $A$ and $B$ at clause $C_{1}\vee C_{2}$.
That means:

\begin{eqnarray}
A\wedge\neg C_{1}\backslash B\wedge\neg C_{2}\backslash B & \models & I\label{eq:9}\\
B\wedge\neg C_{1}\downarrow B\wedge\neg C_{2}\downarrow B\wedge I & \models & \bot\label{eq:10}
\end{eqnarray}



\subsubsection*{Proof of (\ref{eq:9}):}

The inductiveness of the partial interpolants of the premises gives
us:

\begin{eqnarray}
A\wedge\neg C_{1}\backslash B\wedge t_{A}-y\geq0 & \models & I_{l}\label{eq:11}\\
A\wedge\neg C_{2}\backslash B\wedge t_{A}-y\leq0 & \models & I_{r}\label{eq:12}
\end{eqnarray}


In the following, assume $A\wedge\neg C_{1}\backslash B\wedge\neg C_{2}\backslash B$.

By instantiating $y$ with $t_{A}$ in (\ref{eq:11}) and in (\ref{eq:12}),
we get $\models I_{1}\left[EQ_{R}^{\left(1\right)}\left(t_{A}\right)\right]$
and $\models I_{2}\left[EQ_{R}^{\left(2\right)}\left(t_{A}\right)\right]$.

For proving (\ref{eq:9}), it remains to be shown that $I_{1}\left[EQ_{R}^{\left(1\right)}\left(t_{A}\right)\right]\wedge I_{2}\left[EQ_{R}^{\left(2\right)}\left(t_{A}\right)\right]\models I_{1}\left[I_{2}\left[EQ_{R}^{\left(3\right)}\right]\right]$.
\begin{enumerate}
\item We begin by proving $\models EQ_{R}^{\left(1\right)}\left(t_{A}\right)\wedge EQ_{R}^{\left(2\right)}\left(t_{A}\right)\rightarrow EQ_{R}^{\left(3\right)}$:


simple transformations give us:


\begin{eqnarray*}
 &  & EQ_{R}^{\left(1\right)}\left(t_{A}\right)\wedge EQ_{R}^{\left(2\right)}\left(t_{A}\right)\\
 & \equiv & \frac{s_{2}}{c_{2}}\leq t_{A}\wedge t_{A}\leq-\frac{s_{1}}{c_{1}}\\
 &  & \wedge\left(t_{A}<-\frac{s_{1}}{c_{1}}\vee F_{1}\left(t_{A}\right)\right)\wedge\left(\frac{s_{2}}{c_{2}}<t_{A}\vee F_{2}\left(t_{A}\right)\right)
\end{eqnarray*}



and


\begin{eqnarray*}
 &  & EQ_{R}^{\left(3\right)}\\
 & \equiv & \frac{s_{2}}{c_{2}}\leq-\frac{s_{1}}{c_{1}}\wedge\left(\frac{s_{2}}{c_{2}}<-\frac{s_{1}}{c_{1}}\vee\left(F_{1}\left(\frac{s_{2}}{c_{2}}\right)\wedge F_{2}\left(\frac{s_{2}}{c_{2}}\right)\right)\right)
\end{eqnarray*}



Case split on $\frac{s_{2}}{c_{2}}=-\frac{s_{1}}{c_{1}}$:


It holds that $EQ_{R}^{\left(1\right)}\left(t_{A}\right)\wedge EQ_{R}^{\left(2\right)}\left(t_{A}\right)\implies\frac{s_{2}}{c_{2}}\leq t_{A}\wedge t_{A}\leq-\frac{s_{1}}{c_{1}}\implies\frac{s_{2}}{c_{2}}\leq-\frac{s_{1}}{c_{1}}$.({*}) 
\begin{itemize}
\item Case ($\frac{s_{2}}{c_{2}}\neq-\frac{s_{1}}{c_{1}}$):


From ({*}) together with $\frac{s_{2}}{c_{2}}\neq-\frac{s_{1}}{c_{1}}$
we get $\frac{s_{2}}{c_{2}}<-\frac{s_{1}}{c_{1}}$ which means that
$EQ_{R}^{\left(3\right)}$ holds.

\item Case ($\frac{s_{2}}{c_{2}}=-\frac{s_{1}}{c_{1}}$):


By $\frac{s_{2}}{c_{2}}\leq t_{A}\wedge t_{A}\leq-\frac{s_{1}}{c_{1}}$
(from ({*})) and $\frac{s_{2}}{c_{2}}=-\frac{s_{1}}{c_{1}}$, we get
$t_{A}=\frac{s_{2}}{c_{2}}=-\frac{s_{1}}{c_{1}}$. So for $EQ_{R}^{\left(1\right)}\wedge EQ_{R}^{\left(2\right)}$
to hold, $F_{1}\left(\frac{s_{2}}{c_{2}}\right)\wedge F_{2}\left(\frac{s_{2}}{c_{2}}\right)$
must hold which gives the second conjunct of $EQ_{R}^{\left(3\right)}$,
while ({*}) gives the first.

\end{itemize}
\item From $\models EQ_{R}^{\left(1\right)}\left(t_{A}\right)\wedge EQ_{R}^{\left(2\right)}\left(t_{A}\right)\rightarrow EQ_{R}^{\left(3\right)}$
together with monotonicity we get our proof goal:


\begin{eqnarray*}
 &  & EQ_{R}^{\left(1\right)}\left(t_{A}\right)\wedge EQ_{R}^{\left(2\right)}\left(t_{A}\right)\rightarrow EQ_{R}^{\left(3\right)}\\
 & \equiv & EQ_{R}^{\left(1\right)}\left(t_{A}\right)\rightarrow\left(EQ_{R}^{\left(2\right)}\left(t_{A}\right)\rightarrow EQ_{R}^{\left(3\right)}\right)\\
 & \implies & EQ_{R}^{\left(1\right)}\rightarrow\left(I_{2}\left[EQ_{R}^{\left(2\right)}\left(t_{A}\right)\right]\rightarrow I_{2}\left[EQ_{R}^{\left(3\right)}\right]\right)\\
 & \equiv & EQ_{R}^{\left(1\right)}\left(t_{A}\right)\wedge I_{2}\left[EQ_{R}^{\left(2\right)}\left(t_{A}\right)\right]\rightarrow I_{2}\left[EQ_{R}^{\left(3\right)}\right]\\
 & \equiv & I_{2}\left[EQ_{R}^{\left(2\right)}\left(t_{A}\right)\right]\rightarrow\left(EQ_{R}^{\left(1\right)}\left(t_{A}\right)\rightarrow I_{2}\left[EQ_{R}^{\left(3\right)}\right]\right)\\
 & \implies & I_{2}\left[EQ_{R}^{\left(2\right)}\left(t_{A}\right)\right]\rightarrow\left(I_{1}\left[EQ_{R}^{\left(1\right)}\left(t_{A}\right)\right]\rightarrow I_{1}\left[I_{2}\left[EQ_{R}^{\left(3\right)}\right]\right]\right)\\
 & \implies & I_{2}\left[EQ_{R}^{\left(2\right)}\left(t_{A}\right)\right]\wedge I_{1}\left[EQ_{R}^{\left(1\right)}\left(t_{A}\right)\right]\rightarrow I_{1}\left[I_{2}\left[EQ_{R}^{\left(3\right)}\right]\right]
\end{eqnarray*}


\end{enumerate}



\subsubsection*{Proof of (\ref{eq:10}):}

By the contradictory property of the interpolants $I_{l}$ and $I_{r}$
we get:

\begin{eqnarray}
A\wedge\neg C_{1}\downarrow B\wedge y+t_{B}>0\wedge I_{l} & \models & \bot\label{eq:13}\\
A\wedge\neg C_{2}\downarrow B\wedge y+t_{B}\leq0\wedge I_{r} & \models & \bot\label{eq:14}
\end{eqnarray}


Assume $B\wedge\neg C_{1}\downarrow B\wedge\neg C_{2}\downarrow B$.
\begin{enumerate}
\item In \ref{eq:14}, we instantiate $y:=-\frac{s_{1}}{c_{1}}$. With the
assumptions omitted we get:


\begin{eqnarray*}
 &  & -\frac{s_{1}}{c_{1}}+t_{B}\leq0\wedge I_{r}\rightarrow\bot\\
 & \implies & I_{2}\left[EQ_{R}^{\left(2\right)}\left(-\frac{s_{1}}{c_{1}}\right)\right]\rightarrow t_{B}>\frac{s_{1}}{c_{1}}\\
 & \equiv & I_{2}\left[EQ_{R}\left(\frac{c_{2}s_{1}}{c_{1}}+s_{2},F_{2}\left(-\frac{s_{1}}{c_{1}}\right)\right)\right]\rightarrow t_{B}>\frac{s_{1}}{c_{1}}\\
 & \equiv & I_{2}\left[\frac{c_{2}s_{1}}{c_{1}}+s_{2}\leq0\wedge\left(\frac{c_{2}s_{1}}{c_{1}}+s_{2}<0\vee F_{2}\left(-\frac{s_{1}}{c_{1}}\right)\right)\right]\rightarrow t_{B}>\frac{s_{1}}{c_{1}}\\
 & \equiv & I_{2}\left[c_{2}s_{1}+c_{1}s_{2}\leq0\wedge\left(c_{2}s_{1}+c_{1}s_{2}<0\vee F_{2}\left(-\frac{s_{1}}{c_{1}}\right)\right)\right]\rightarrow t_{B}>\frac{s_{1}}{c_{1}}\\
 & \implies & I_{2}\left[EQ_{R}^{\left(3\right)}\right]\rightarrow t_{B}>\frac{s_{1}}{c_{1}}
\end{eqnarray*}



(The last transformation holds because $EQ_{R}^{\left(3\right)}$
occurs only positive in $I_{2}$ and thus $I_{2}\left[EQ_{R}^{\left(3\right)}\right]\implies I_{2}\left[c_{2}s_{1}+c_{1}s_{2}\leq0\wedge\left(c_{2}s_{1}+c_{1}s_{2}<0\vee F_{2}\left(-\frac{s_{1}}{c_{1}}\right)\right)\right]$.
Then we get the last line by transitivity.)

\item Next, we instantiate $y:=-\frac{t_{B}+\nicefrac{s_{1}}{c_{1}}}{2}$
in \ref{eq:13} and we get: (for brevity delay the instantiation for
the first lines..)


\begin{eqnarray*}
 &  & y+t_{B}\wedge I_{l}\rightarrow\bot\\
 & \implies & I_{l}\rightarrow y+t_{B}\leq0\\
 & \equiv & I_{1}\left[c_{1}y+s_{1}\wedge\left(c_{1}y+s_{1}<0\vee F_{1}\left(y\right)\right)\right]\rightarrow y+t_{B}\leq0\\
(inst\, y) & \equiv & I_{1}\left[t_{B}\geq\frac{s_{1}}{c_{1}}\wedge\left(t_{B}>\frac{s_{1}}{c_{1}}\vee F_{1}\left(-\frac{t_{B}+\nicefrac{s_{1}}{c_{1}}}{2}\right)\right)\right]\rightarrow t_{B}\leq\frac{s_{1}}{c_{1}}
\end{eqnarray*}


\item Next, we do a case split on $t_{B}>\frac{s_{1}}{c_{1}}$:

\begin{itemize}
\item case $t_{B}>\frac{s_{1}}{c_{1}}$:


Then $I_{1}\left[t_{B}\geq\frac{s_{1}}{c_{1}}\wedge\left(t_{B}>\frac{s_{1}}{c_{1}}\vee F_{1}\left(-\frac{t_{B}+\nicefrac{s_{1}}{c_{1}}}{2}\right)\right)\right]\rightarrow\bot$
and because $I_{2}\left[EQ_{R}^{\left(3\right)}\right]\rightarrow t_{B}>\frac{s_{1}}{c_{1}}$
(from 1.) and $t_{B}>\frac{s_{1}}{c_{1}}\rightarrow\left(t_{B}\geq\frac{s_{1}}{c_{1}}\wedge\left(t_{B}>\frac{s_{1}}{c_{1}}\vee F_{1}\left(-\frac{t_{B}+\nicefrac{s_{1}}{c_{1}}}{2}\right)\right)\right)$,
we can conclude that $I_{1}\left[I_{2}\left[EQ_{R}^{\left(3\right)}\right]\right]\rightarrow\bot$
and are done.

\item case $t_{B}\leq\frac{s_{1}}{c_{1}}$:


From instantiating $y:=-t_{B}+1$ in (\ref{eq:13}) we get:


\begin{eqnarray*}
 &  & I_{l}\rightarrow-t_{B}+1+t_{B}\leq0\\
 & \equiv & I_{1}\left[c_{1}\left(-t_{B}+1\right)+s_{1}\leq0\wedge...\right]\rightarrow\bot\\
 & \equiv & I_{1}\left[t_{B}+1\geq\frac{s_{1}}{c_{1}}\wedge...\right]\rightarrow\bot\\
\text{(case \ensuremath{t_{B}\leq\frac{s_{1}}{c_{1}}})} & \equiv & I_{1}\left[\bot\right]\rightarrow\bot
\end{eqnarray*}



We still know from 1. that $I_{2}\left[EQ_{R}^{\left(3\right)}\right]\rightarrow t_{B}>\frac{s_{1}}{c_{1}}$,
so because we are in case $\ensuremath{t_{B}\leq\frac{s_{1}}{c_{1}}}$
this means that $I_{2}\left[EQ_{R}^{\left(3\right)}\right]\rightarrow\bot$.
So, we can conclude that $I_{1}\left[I_{2}\left[EQ_{R}^{\left(3\right)}\right]\right]\rightarrow\bot$
.

\end{itemize}
\end{enumerate}



\section{Mixed Integer Inequalities}

The restriction of the inequalities to $A$ and $B$ are defined the
same way as for the rational inequalities.


\subsubsection*{Definitions:}

\begin{eqnarray*}
F & :\equiv & \exists x.(\frac{s_{2}}{c_{2}}\leq x\wedge x\leq\frac{s_{2}}{c_{2}}+\frac{k_{1}}{c_{1}}+\frac{k_{2}}{c_{2}}\\
 &  & \wedge\left(c_{1}x+s_{1}\leq-k_{1}\vee F_{1}\left(x\right)\right)\wedge\left(-c_{2}x+s_{2}\leq-k_{2}\vee F_{2}\left(x\right)\right))\\
EQ_{Z}\left(s,k,F\right) & :\equiv & s\leq0\wedge\left(s\leq-k\vee F\right)
\end{eqnarray*}


The pivoting rule for pivot element $t\leq0$ is:

\[\inferrule {t\leq 0 \vee C_1 : I_1[EQ_Z (c_1y + s_1, k_1, F_1(y)] \\ t > 0 \vee C_2: I_2[EQ_Z (-c_2y + s_2, k_2, F(y)} {C_1 \vee C_2: I_1[I_2[EQ_Z(c_2 s_1 + c_1 s_2, c_2 k_1 + c_1 k_2 + (c_1 - 1)(c_2 - 1), F)] } \]


\subsubsection*{Claim:}

$I:\equiv I_{1}\left[I_{2}\left[EQ_{Z}(c_{2}s_{1}+c_{1}s_{2},c_{2}k_{1}+c_{1}k_{2}+c_{1}c_{2},F)\right]\right]$
is a partial interpolant for $A$ and $B$ at resolution clause $C_{1}\vee C_{2}$.
That means:

\begin{eqnarray}
A\wedge\neg C_{1}\backslash B\wedge\neg C_{2}\backslash B & \models & I\label{eq:15}\\
B\wedge\neg C_{1}\downarrow B\wedge\neg C_{2}\downarrow B\wedge I & \models & \bot\label{eq:16}
\end{eqnarray}



\subsubsection*{Definitions/Shorthands:}

\begin{eqnarray*}
EQ_{Z}^{\left(1\right)}\left(y\right) & :\equiv & EQ_{Z}\left(c_{1}y+s_{1},k_{1},F_{1}(y)\right)\\
EQ_{Z}^{\left(2\right)}\left(y\right) & :\equiv & EQ_{Z}\left(-c_{2}y+s_{2},k_{2},F_{2}(y)\right)\\
EQ_{Z}^{\left(3\right)} & :\equiv & EQ_{Z}\left(c_{2}s_{1}+c_{1}s_{2},c_{2}k_{1}+c_{1}k_{2}+(c_{1}-1)(c_{2}-1),F\right)\\
I_{l} & :\equiv & I_{1}\left[EQ_{Z}^{\left(1\right)}\left(y\right)\right]\\
I_{r} & :\equiv & I_{2}\left[EQ_{Z}^{\left(2\right)}\left(y\right)\right]\\
I & :\equiv & I_{1}\left[I_{2}\left[EQ_{Z}^{\left(3\right)}\right]\right]
\end{eqnarray*}



\subsubsection*{Proof of \ref{eq:15}:}

We know that $I_{l}$ and $I_{r}$ are partial interpolants:

\begin{eqnarray}
A\wedge\neg C_{1}\backslash B\wedge t_{A}-y\geq0 & \models & I_{l}\label{eq:17}\\
A\wedge\neg C_{2}\backslash B\wedge t_{A}-y\leq0 & \models & I_{r}\label{eq:18}
\end{eqnarray}


Assume $A\wedge C_{1}\backslash B\wedge C_{2}\backslash B$.
\begin{enumerate}
\item As for the proof for rationals, we show that $\exists y.EQ_{Z}^{\left(1\right)}\left(y\right)\wedge EQ_{Z}^{\left(2\right)}\left(y\right)\rightarrow EQ_{Z}^{\left(3\right)}$
and conclude the proof goal by monotonicity.


Let 
\begin{eqnarray*}
\left(1\right): & = & c_{1}y+s_{1}\leq0\\
\left(1.1\right) & := & c_{1}y+s_{1}\leq-k_{1}\\
\left(1.2\right) & := & F_{1}\left(y\right)
\end{eqnarray*}



Then $EQ_{Z}^{\left(1\right)}=\left(1\right)\wedge\left(\left(1.1\right)\vee\left(1.2\right)\right)$
and analogously $EQ_{Z}^{\left(2\right)}=\left(2\right)\wedge\left(\left(2.1\right)\vee\left(2.2\right)\right)$.


Assuming $\exists y.EQ_{Z}^{\left(1\right)}\left(y\right)\wedge EQ_{Z}^{\left(2\right)}\left(y\right)$
we want to proof


\begin{eqnarray*}
EQ_{Z}^{\left(3\right)} & = & c_{2}s_{1}+c_{1}s_{2}\leq0\\
 & \wedge & (c_{2}s_{1}+c_{1}s_{2}\leq-c_{2}k_{1}-c_{1}k_{2}-(c_{1}-1)(c_{2}-1)\\
 & \vee & \exists x.(\frac{s_{2}}{c_{2}}\leq x\wedge x\leq\frac{s_{2}}{c_{2}}+\frac{k_{1}}{c_{1}}+\frac{k_{2}}{c_{2}}+\frac{(c_{1}-1)(c_{2}-1)}{c_{1}c_{2}}\\
 &  & \wedge\left(c_{1}x+s_{1}\leq-k_{1}\vee F_{1}\left(x\right)\right)\wedge\left(c_{2}x+s_{2}\leq-k_{2}\vee F_{2}\left(x\right)\right))
\end{eqnarray*}



``$c_{2}s_{1}+c_{1}s_{2}\leq0$'', the first entry of the big conjunction,
follows directly from $\left(2\right)\wedge\left(1\right)\equiv\frac{s_{2}}{c_{2}}\leq y\wedge y\leq-\frac{s_{1}}{c_{1}}$
(putting the according constraint on the existentially quantified
$y$ we have to choose in order to to fulfill the rest of the formula).


Next, we have to show that any pair from $p\in\left\{ \left(1.1\right),\left(1.2\right)\right\} \times\left\{ \left(2.1\right),\left(2.2\right)\right\} $
(possibly together with $\left(1\right)\wedge\left(2\right)$) implies
the rest of $EQ_{Z}^{\left(3\right)}$.


In the following assume that $\neg\left(c_{2}s_{1}+c_{1}s_{2}\leq-c_{2}k_{1}-c_{1}k_{2}-(c_{1}-1)(c_{2}-1)\right)$
is true and show that under this assumption $F$ has to hold.


Split $F$ in three parts such that $F=\exists x.\left(F_{a}\wedge F_{b}\wedge F_{c}\right)$: 


\begin{eqnarray*}
F_{a} & := & \frac{s_{2}}{c_{2}}\leq x\wedge x\leq\frac{s_{2}}{c_{2}}+\frac{k_{1}}{c_{1}}+\frac{k_{2}}{c_{2}}+\frac{(c_{1}-1)(c_{2}-1)}{c_{1}c_{2}}\\
F_{b} & := & \left(c_{1}x+s_{1}\leq-k_{1}\vee F_{1}\left(x\right)\right)\\
F_{c} & := & \left(c_{2}x+s_{2}\leq-k_{2}\vee F_{2}\left(x\right)\right)
\end{eqnarray*}



choosing $x:=y$, clearly $F_{b}$ and $F_{c}$ follow from each pair
$p$, $F_{a}$ remains to be shown. 


From $x=y$ and the constraint on $y$ we get $\frac{s_{2}}{c_{2}}\leq x$
which is the lower bound we need. We also get $x\leq-\frac{s_{1}}{c_{1}}$.


From the assumption $\neg\left(c_{2}s_{1}+c_{1}s_{2}\leq-c_{2}k_{1}-c_{1}k_{2}-(c_{1}-1)(c_{2}-1)\right)$
we get:


\begin{eqnarray*}
 &  & c_{2}s_{1}+c_{1}s_{2}>-c_{2}k_{1}-c_{1}k_{2}-(c_{1}-1)(c_{2}-1)\\
 & \equiv & -\frac{s_{1}}{c_{1}}<\frac{s_{2}}{c_{2}}+\frac{k_{1}}{c_{1}}+\frac{k_{2}}{c_{2}}+\frac{(c_{1}-1)(c_{2}-1)}{c_{1}c_{2}}
\end{eqnarray*}



we get $x\leq-\frac{s_{1}}{c_{1}}<\frac{s_{2}}{c_{2}}+\frac{k_{1}}{c_{1}}+\frac{k_{2}}{c_{2}}+\frac{(c_{1}-1)(c_{2}-1)}{c_{1}c_{2}}$
are done.

\item We have shown $\exists y.EQ_{Z}^{\left(1\right)}\left(y\right)\wedge EQ_{Z}^{\left(2\right)}\left(y\right)\rightarrow EQ_{Z}^{\left(3\right)}$.
The rest of the proof works completely analogous to the rational case.
\end{enumerate}



\subsubsection*{Proof of (\ref{eq:16}):}

We know that $I_{r}$ and $I_{l}$ are partial interpolants:

\begin{eqnarray}
B\wedge C_{1}\downarrow B\wedge y-t_{B}>0\wedge I_{l} & \models & \bot\label{eq:19}\\
B\wedge C_{2}\downarrow B\wedge y-t_{B}\leq0\wedge I_{r} & \models & \bot\label{eq:20}
\end{eqnarray}


Assume $B\wedge C_{1}\downarrow B\wedge C_{2}\downarrow B$.

From (\ref{eq:19}) and (\ref{eq:20}) 

21:
\begin{eqnarray*}
y-t_{B}>0\wedge I_{1}\left[EQ_{Z}\left(c_{1}y+s_{1},k_{1},F_{1}(y)\right)\right]\rightarrow\bot\\
I_{1}\left[EQ_{Z}^{\left(1\right)}\left(y\right)\right]\rightarrow y-t_{B}\leq0 & \left(23\right)\\
I_{1}\left[EQ_{Z}\left(c_{1}y+s_{1},k_{1},F_{1}(y)\right)\right]\rightarrow y-t_{B}\leq0
\end{eqnarray*}


22:

\begin{eqnarray*}
y-t_{B}\leq0\wedge I_{2}\left[EQ_{Z}\left(-c_{2}y+s_{2},k_{2},F_{2}(y)\right)\right]\rightarrow\bot\\
I_{2}\left[EQ_{Z}^{\left(2\right)}\left(y\right)\right]\rightarrow y-t_{B}>0 & \left(24\right)\\
I_{2}\left[EQ_{Z}\left(-c_{2}y+s_{2},k_{2},F_{2}(y)\right)\right]\rightarrow y-t_{B}>0
\end{eqnarray*}


------------------------------------ 

Lemma: $EQ_{Z}^{\left(3\right)}\rightarrow\exists y.EQ_{Z}^{\left(1\right)}\left(y\right)\wedge EQ_{Z}^{\left(2\right)}\left(y\right)$

Proof:

Case split on whether $F$ holds or not:

In any case for $EQ_{Z}^{\left(3\right)}$ to hold, $c_{2}s_{1}+c_{1}s_{2}\leq0\equiv\frac{s_{2}}{c_{2}}\leq-\frac{s_{1}}{c_{1}}$
has to hold. ({*})
\begin{itemize}
\item Case $F$ holds:


Let $y$ be the witness for the truth of the existential quantifier
in $F$.


Thus $(\frac{s_{2}}{c_{2}}\leq y\wedge y\leq\frac{s_{2}}{c_{2}}+\frac{k_{1}}{c_{1}}+\frac{k_{2}}{c_{2}}\wedge\left(c_{1}y+s_{1}\leq-k_{1}\vee F_{1}\left(y\right)\right)\wedge\left(-c_{2}y+s_{2}\leq-k_{2}\vee F_{2}\left(y\right)\right))$
holds.


We have to show that $EQ_{Z}^{\left(1\right)}\left(y\right)\wedge EQ_{Z}^{\left(2\right)}\left(y\right)$
holds under this assumption.


\begin{eqnarray*}
 &  & EQ_{Z}^{\left(1\right)}\left(y\right)\wedge EQ_{Z}^{\left(2\right)}\left(y\right)\\
 & \equiv & c_{1}y+s_{1}\leq0\wedge\left(c_{1}y+s_{1}\leq-k_{1}\vee F_{1}\left(y\right)\right)\\
 &  & \wedge-c_{2}y+s_{2}\leq0\wedge\left(-c_{2}y+s_{2}\leq-k_{2}\vee F_{2}\left(y\right)\right)
\end{eqnarray*}



(naming as in the proof before)


(2) is implied by $F_{a}$ (first conjunct).


(1) is implied by $F_{a}$ (second conjunct) and ({*}) and transitivity
and $\frac{k_{i}}{c_{i}}\geq0$.


the rest is clearly implied (syntactically equal) by (1.1),(1.2),(2.1),(2.2).

\item Case $F$ does not hold:


Thus for $EQ_{Z}^{\left(3\right)}$ to hold, $\left(c_{2}s_{1}+c_{1}s_{2}\leq-c_{2}k_{1}-c_{1}k_{2}-(c_{1}-1)(c_{2}-1)\right)$
has to hold.


\begin{eqnarray*}
 &  & c_{2}s_{1}+c_{1}s_{2}\leq-c_{2}k_{1}-c_{1}k_{2}-(c_{1}-1)(c_{2}-1)\\
 & \equiv & \frac{s_{2}+k_{2}}{c_{2}}\leq-\frac{s_{1}}{c_{1}}-\frac{k_{1}}{c_{1}}-\frac{(c_{1}-1)(c_{2}-1)}{c_{1}c_{2}}
\end{eqnarray*}



Claim: $EQ_{Z}^{\left(1\right)}\left(y\right)\wedge EQ_{Z}^{\left(2\right)}\left(y\right)$
is fulfilled by $y=\lceil\frac{s_{2}+k_{2}}{c_{2}}\rceil$.


\begin{eqnarray*}
 &  & EQ_{Z}^{\left(1\right)}\left(\lceil\frac{s_{2}+k_{2}}{c_{2}}\rceil\right)\wedge EQ_{Z}^{\left(2\right)}\left(\lceil\frac{s_{2}+k_{2}}{c_{2}}\rceil\right)\\
 & \equiv & c_{1}\lceil\frac{s_{2}+k_{2}}{c_{2}}\rceil+s_{1}\leq0\wedge\left(c_{1}\lceil\frac{s_{2}+k_{2}}{c_{2}}\rceil+s_{1}\leq-k_{1}\vee F_{1}\left(\lceil\frac{s_{2}+k_{2}}{c_{2}}\rceil\right)\right)\\
 &  & \wedge-c_{2}\lceil\frac{s_{2}+k_{2}}{c_{2}}\rceil+s_{2}\leq0\wedge\left(-c_{2}\lceil\frac{s_{2}+k_{2}}{c_{2}}\rceil+s_{2}\leq-k_{2}\vee F_{2}\left(\lceil\frac{s_{2}+k_{2}}{c_{2}}\rceil\right)\right)\\
 & \equiv & \lceil\frac{s_{2}+k_{2}}{c_{2}}\rceil\leq-\frac{s_{1}}{c_{1}}\wedge\left(\lceil\frac{s_{2}+k_{2}}{c_{2}}\rceil\leq\frac{-k_{1}-s_{1}}{c_{1}}\vee F_{1}\left(\lceil\frac{s_{2}+k_{2}}{c_{2}}\rceil\right)\right)\\
 &  & \wedge\lceil\frac{s_{2}+k_{2}}{c_{2}}\rceil\geq\frac{s_{2}}{c_{2}}\wedge\left(\lceil\frac{s_{2}+k_{2}}{c_{2}}\rceil\geq\frac{k_{2}+s_{2}}{c_{2}}\vee F_{2}\left(\lceil\frac{s_{2}+k_{2}}{c_{2}}\rceil\right)\right)
\end{eqnarray*}



The lowest line clearly holds, and as we do not know anything about
$F_{1}$, and as $k_{1}\geq0$ (meaning that, $\lceil\frac{s_{2}+k_{2}}{c_{2}}\rceil\leq\frac{-k_{1}-s_{1}}{c_{1}}\rightarrow\lceil\frac{s_{2}+k_{2}}{c_{2}}\rceil\leq-\frac{s_{1}}{c_{1}}$),
it remains to be shown that:


\begin{eqnarray*}
 &  & \lceil\frac{s_{2}+k_{2}}{c_{2}}\rceil\leq\frac{-k_{1}-s_{1}}{c_{1}}
\end{eqnarray*}



We start with some transformation:


\begin{eqnarray*}
 &  & \frac{s_{2}+k_{2}}{c_{2}}\leq-\frac{s_{1}}{c_{1}}-\frac{k_{1}}{c_{1}}-\frac{(c_{1}-1)(c_{2}-1)}{c_{1}c_{2}}\\
 & \equiv & \frac{s_{2}+k_{2}}{c_{2}}+\frac{c_{2}-1}{c_{2}}\leq\frac{-k_{1}-s_{1}}{c_{1}}-\frac{(c_{1}-1)(c_{2}-1)}{c_{1}c_{2}}+\frac{c_{2}-1}{c_{2}}\\
 & \equiv & \frac{s_{2}+k_{2}}{c_{2}}+\frac{c_{2}-1}{c_{2}}\leq\frac{-k_{1}-s_{1}}{c_{1}}+\frac{c_{1}\left(c_{2}-1\right)-(c_{1}-1)(c_{2}-1)}{c_{1}c_{2}}\\
 & \equiv & \frac{s_{2}+k_{2}}{c_{2}}+\frac{c_{2}-1}{c_{2}}\leq\frac{-k_{1}-s_{1}}{c_{1}}+\frac{\left(c_{2}-1\right)}{c_{1}c_{2}}\\
 & \equiv & \left\lfloor \frac{s_{2}+k_{2}}{c_{2}}+\frac{c_{2}-1}{c_{2}}\right\rfloor \leq\left\lfloor \frac{-k_{1}-s_{1}}{c_{1}}+\frac{\left(c_{2}-1\right)}{c_{1}c_{2}}\right\rfloor 
\end{eqnarray*}

\begin{itemize}
\item Claim: $\left\lceil \frac{s_{2}+k_{2}}{c_{2}}\right\rceil =\left\lfloor \frac{s_{2}+k_{2}}{c_{2}}+\frac{c_{2}-1}{c_{2}}\right\rfloor $


Let $\frac{s_{2}+k_{2}}{c_{2}}=n+\frac{m}{c_{2}}$ with $n\in\mathbb{Z},\frac{m}{c_{2}}<1$.
Then if $m=0$, $\frac{s_{2}+k_{2}}{c_{2}}$ is a whole number and
$\frac{s_{2}+k_{2}}{c_{2}}=\left\lceil \frac{s_{2}+k_{2}}{c_{2}}\right\rceil =\left\lfloor \frac{s_{2}+k_{2}}{c_{2}}\right\rfloor $and
because $\frac{c_{2}-1}{c_{2}}<1$, the claim holds. Otherwise $1\leq m\leq c_{2}-1$,$\left\lceil \frac{s_{2}+k_{2}}{c_{2}}\right\rceil =n+1$,
and $\left\lfloor \frac{s_{2}+k_{2}}{c_{2}}+\frac{c_{2}-1}{c_{2}}\right\rfloor =\left\lfloor n+\frac{m}{c_{2}}+\frac{c_{2}-1}{c_{2}}\right\rfloor =n+1$.

\end{itemize}

So together with the claim and the inequality directly above we get
$\left\lceil \frac{s_{2}+k_{2}}{c_{2}}\right\rceil \leq\left\lfloor \frac{-k_{1}-s_{1}}{c_{1}}+\frac{\left(c_{2}-1\right)}{c_{1}c_{2}}\right\rfloor $.


Furthermore $\left\lfloor \frac{-k_{1}-s_{1}}{c_{1}}+\frac{\left(c_{2}-1\right)}{c_{1}c_{2}}\right\rfloor \leq\frac{-s_{1}-k_{1}}{c_{1}}$
holds, because $\frac{\left(c_{2}-1\right)}{c_{1}c_{2}}<\frac{1}{c_{1}}$.
(From a fraction with denominator $x$, we cannot reach the next bigger
whole number by adding something lower than $\frac{1}{x}$.)


From these two statements, we can conclude our proof goal $\left\lceil \frac{s_{2}+k_{2}}{c_{2}}\right\rceil \leq\frac{-s_{1}-k_{1}}{c_{1}}$.

\end{itemize}
------------------------------------

For the main proof, we make a case split on whether $EQ_{Z}^{\left(3\right)}$
holds or not. 
\begin{itemize}
\item Case $EQ_{Z}^{\left(3\right)}$ holds


Then from the Lemma, we know that $\exists y.\left(EQ_{Z}^{\left(1\right)}\left(y\right)\wedge EQ_{Z}^{\left(2\right)}\left(y\right)\right)$
holds, too.


So for some $y$ $EQ_{Z}^{\left(1\right)}\left(y\right)\wedge EQ_{Z}^{\left(2\right)}\left(y\right)$
holds. Fix that $y$.


We make a second case split on $y-t_{B}>0$:
\begin{itemize}
\item Case $y-t_{B}>0$


In this case, we get from (23) that $I_{1}\left[EQ_{Z}^{\left(1\right)}\left(y\right)\right]\rightarrow\bot$.
({*})


From the lemma and the ``big'' case split we know that $EQ_{Z}^{\left(1\right)}\equiv\top$.


Clearly $I_{2}\left[EQ_{Z}^{\left(3\right)}\right]\rightarrow\top$,
thus $I_{2}\left[EQ_{Z}^{\left(3\right)}\right]\rightarrow EQ_{Z}^{\left(1\right)}$.


From this and ({*}) and that the parameters of the interpolants occur
only positive, we conclude our proof goal $I_{1}\left[I_{2}\left[EQ_{Z}^{\left(3\right)}\right]\right]\rightarrow\bot$. 

\item Case $y-t_{B}\leq0$


In this case, (24) gives us $I_{2}\left[EQ_{Z}^{\left(2\right)}\left(y\right)\right]\rightarrow\bot$.
By monotonicity we get $I_{1}\left[I_{2}\left[EQ_{Z}^{\left(2\right)}\left(y\right)\right]\right]\rightarrow I_{1}\left[\bot\right]$.


From instantiating $y:=t_{B}+1$ in (23), we get $I_{1}\left[EQ_{Z}^{\left(1\right)}\left(t_{B}+1\right)\right]\rightarrow\bot$,
as $\bot\rightarrow EQ_{Z}^{\left(1\right)}\left(t_{B}+1\right)$,
we conclude that $I_{1}\left[\bot\right]\rightarrow\bot$ (again bc
of only positive occurence of interpolant parameters), and by transitivity
of ``$\rightarrow$'', we get the proof goal $I_{1}\left[I_{2}\left[EQ_{Z}^{\left(2\right)}\left(y\right)\right]\right]\rightarrow\bot$

\end{itemize}
\end{itemize}

\begin{itemize}
\item Case $EQ_{Z}^{\left(3\right)}$ does not hold:


By instantiating $y:=t_{B}$ in (24), we get:


\[
I_{2}\left[EQ_{Z}^{\left(2\right)}\left(t_{B}\right)\right]\rightarrow\bot
\]



By \emph{ex falso quod libet}, we get $EQ_{Z}^{\left(3\right)}\rightarrow EQ_{Z}^{\left(2\right)}\left(t_{B}\right)$
and thus
\[
I_{2}\left[EQ_{Z}^{\left(3\right)}\right]\rightarrow\bot
\]



By instantiating $y:=t_{B}+1$ in (23), we get:


\[
I_{1}\left[EQ_{Z}^{\left(1\right)}\left(t_{B}+1\right)\right]\rightarrow\bot
\]



Again using \emph{ex falso quod libet}, we get $I_{2}\left[EQ_{Z}^{\left(3\right)}\right]\rightarrow EQ_{Z}^{\left(1\right)}\left(t_{B}+1\right)$
and thus our proof goal $I_{1}\left[I_{2}\left[EQ_{Z}^{\left(3\right)}\right]\right]\rightarrow\bot$. 

\end{itemize}

\end{document}
